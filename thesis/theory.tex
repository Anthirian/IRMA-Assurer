\section{Theory}
\subsection{IRMA}
Here we explain the background of the IRMA proofs.

\subsubsection{Usage scenario}
Beschrijven hoe de basic course of events verloopt, en welke voorwaarden daarbij gelden. Denk aan het veilig opbergen van tablet in kluis, wellicht pincode op tablet, etc.

\subsection{Passport}
The passport is a travel document issued by a country's government that certifies the identity and nationality of its holder for the purpose of international travel~\cite{passportdefinition}. A biometric passport is an upgraded version that contains biometric information that can be used to authenticate the identity of travelers. This information is stored on an embedded chip that can be read using contactless smart card technology. Passports that feature such a chip generally feature the symbol shown in figure~\ref{fig:biometricslogo} on the cover.

\begin{figure}[htb]
	\centering
		\includegraphics{images/biometrics_logo.svg}
	\caption{Symbol to indicate a biometric passport}
	\label{fig:biometricslogo}
\end{figure}

The data stored on the passport chip needs to be protected from modification, cloning, etc. For this purpose several protection mechanisms have been implemented. Each of these mechanisms exist alongside each other and protect against different types of attacks. The most common mechanism is Basic Access Control (BAC). 

\ldots

\textsc{Copied from Wikipedia: \url{https://en.wikipedia.org/wiki/Biometric_passport}} 

Biometric passports are equipped with protection mechanisms to avoid and/or detect attacks:

\begin{itemize}
	\item Non-traceable chip characteristics. Random chip identifiers reply to each request with a different chip number. This prevents tracing of passport chips. Using random identification numbers is optional.
  \item Basic Access Control (BAC). BAC protects the communication channel between the chip and the reader by encrypting transmitted information. Before data can be read from a chip, the reader needs to provide a key which is derived from the Machine Readable Zone: the date of birth, the date of expiry and the document number. If BAC is used, an attacker cannot (easily) eavesdrop transferred information without knowing the correct key. Using BAC is optional.
  \item Passive Authentication (PA). PA is aimed at identifying modification of passport chip data. The chip contains a file (SOD) that stores hash values of all files stored in the chip (picture, fingerprint, etc.) and a digital signature of these hashes. The digital signature is made using a document signing key which itself is signed by a country signing key. If a file in the chip (e.g. the picture) is changed, this can be detected since the hash value is incorrect. Readers need access to all used public country keys to check whether the digital signature is generated by a trusted country. Using PA is mandatory.[citation needed] According to a September 2011 United States Central Intelligence Agency document released by Wikileaks in December 2014, "Although falsified e-passports will not have the correct digital signature, inspectors may not detect the fraud if the passports are from countries that do not participate in the International Civil Aviation Organization’s Public Key Directory (ICAO PKD). Only 15 of over 60 e-passport-issuing countries belong to the PKD program, as of December 2010"[4]
  \item Active Authentication (AA). AA prevents cloning of passport chips. The chip contains a private key that cannot be read or copied, but its existence can easily be proven. Using AA is optional.
  \item Extended Access Control (EAC). EAC adds functionality to check the authenticity of both the chip (chip authentication) and the reader (terminal authentication). Furthermore it uses stronger encryption than BAC. EAC is typically used to protect fingerprints and iris scans. Using EAC is optional. In the European Union, using EAC is mandatory for all documents issued starting 28 June 2009.[citation needed]
  \item Supplemental Access Control (SAC) was introduced by ICAO in 2009 for addressing BAC weaknesses. It was introduced as a supplement to BAC (for keeping compatibility), but will replace it in the future.
  \item Shielding the chip. This prevents unauthorized reading. Some countries – including at least the US – have integrated a very thin metal mesh into the passport's cover to act as a shield when the passport cover is closed.[5] The use of shielding is optional.
\end{itemize}


Since the introduction of biometric passports several attacks have been presented and demonstrated:

\begin{itemize}
	\item Non-traceable chip characteristics. In 2008 a Radboud/Lausitz University team demonstrated that it's possible to determine which country a passport chip is from without knowing the key required for reading it.[7] The team fingerprinted error messages of passport chips from different countries. The resulting lookup table allows an attacker to determine from where a chip originated. In 2010 Tom Chothia and Vitaliy Smirnov documented an attack that allows an individual passport to be traced,[8][9] by sending specific BAC authentication requests.
  \item Basic Access Control (BAC). In 2005 Marc Witteman showed that the document numbers of Dutch passports were predictable,[10] allowing an attacker to guess/crack the key required for reading the chip. In 2006 Adam Laurie wrote software that tries all known passport keys within a given range, thus implementing one of Witteman's attacks. Using online flight booking sites, flight coupons and other public information it's possible to significantly reduce the number of possible keys. Laurie demonstrated the attack by reading the passport chip of a Daily Mail's reporter in its envelope without opening it.[11] Note that in some early biometric passports BAC wasn't used at all, allowing attacker to read the chip's content without providing a key.[12]
  \item Passive Authentication (PA). In 2006 Lukas Grunwald demonstrated that it is trivial to copy passport data from a passport chip into a standard ISO/IEC 14443 smartcard using a standard contactless card interface and a simple file transfer tool.[13] Grunwald used a passport that did not use Active Authentication (anti-cloning) and did not change the data held on the copied chip, thus keeping its cryptographic signature valid. In 2008 Jeroen van Beek demonstrated that not all passport inspection systems check the cryptographic signature of a passport chip. For his demonstration Van Beek altered chip information and signed it using his own document signing key of a non-existing country. This can only be detected by checking the country signing keys that are used to sign the document signing keys. To check country signing keys the ICAO PKD[14] can be used. Only 5 out of 60+ countries are using this central database.[15] Van Beek did not update the original passport chip: instead an ePassport emulator was used.[16] Also in 2008, The Hacker's Choice implemented all attacks and published code to verify the results.[17] The release included a video clip that demonstrated problems by using a forged Elvis Presley passport that is recognized as a valid US passport.[18][19]
  \item Active Authentication (AA). In 2005 Marc Witteman showed that the secret Active Authentication key can be retrieved using power analysis.[10] This may allow an attacker to clone passport chips that use the optional Active Authentication anti-cloning mechanism on chips – if the chip design is susceptible to this attack. In 2008 Jeroen van Beek demonstrated that optional security mechanisms can be disabled by removing their presence from the passport index file.[20] This allows an attacker to remove – amongst others – anti-cloning mechanisms (Active Authentication). The attack is documented in supplement 7 of Doc 9303 (R1-p1\_v2\_sIV\_0006)[21] and can be solved by patching inspection system software. Note that supplement 7 features vulnerable examples in the same document that – when implemented – result in a vulnerable inspection process.
  \item Extended Access Control (EAC). In 2007 Luks Grunwald presented an attack that can make EAC-enabled passport chips unusable.[22] Grunwald states that if an EAC-key – required for reading fingerprints and updating certificates – is stolen or compromised, an attacker can upload a false certificate with an issue date far in the future. The affected chips block read access until the future date is reached.
\end{itemize}



Example: active authentication in e-passport
\begin{itemize}
	\item private key securely embedded in passport chip
  \item public key signed by producer (Morpho in NL)
  \item Morpho's public key signed by Dutch state
\end{itemize}
