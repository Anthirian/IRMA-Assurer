\chapter{Introduction}
Consider a scenario where you have to legitimize yourself in order to purchase a particular item such as cigarettes. The cashier is legally required to check your identification to ensure you are not underage before allowing you to purchase the items. Generally the customer presents some form of identification to the cashier, such as a passport. The cashier then checks your date of birth to ensure you are of legal age. In this process the cashier may also learn your name, location of birth and the document number, among other things. To some this feels like a breach of privacy and these individuals may wish to share only their date of birth, but ideally just the property `of legal age' with the cashier. This is what the IRMA card is designed to do. The name is an acronym of I Reveal My Attributes and it is designed in such a way that you can selectively disclose properties of yourself to select individuals. Before you can use such a card you would need to add properties about your person to it, called attributes, which can then be selectively disclosed to gain various privileges, turning the attributes into attribute based credentials or ABCs. A particularly good set of starting properties to use for generating ABCs are the data stored in a person's passport. For this reason, we have designed the Assurer protocol.

The IRMA card has a lot of potential. There are many use cases the card could improve upon. Assurer could potentially allow for a much quicker adaptation of the IRMA card. The reason for this is that most people already have an identity card or a passport that contains all their characteristics in one place. Leveraging this in order to quickly gather many ABCs at once would be an ideal way to allow newcomers to get started. The passport or identity card generally holds the most generic characteristics, such as name, birthday, height and sex, making its conversion into attributes have immediate visual result.

Assurer is a protocol designed to facilitate transfer these passport characteristics to IRMA cards. In this paper we describe the steps taken during the design process and provide proof it satisfies the goals set out for it. In this protocol we identify three actors: the assurer, the verification and issuing server and the IRMA card. We will describe these further in chapter~\ref{sec:theory}. The protocol is used to convert personally identifiable information stored on a passport into attribute based credentials that can be stored on an IRMA card. This is done by first verifying the passport both locally and remotely, upon which a set of attribute based credentials is generated for storing on the IRMA card. We will explain the protocol in much more detail in chapter~\ref{subsec:bcoe}.

Over the course of this paper we use many terms the reader may not be familiar with. Furthermore abstraction is sometimes favorable, thus we now describe the meaning of certain terms that are used throughout this paper. The IRMA Verified Assurer protocol, denoted `Assurer' from this point onwards, makes use of tablets for the clients that connect to the server. We will simply identify these tablets as `clients' and similarly will refer to the IRMA card simply as `card.' The clients connect to a central server which provides the clients with cryptographically signed attribute data, which we will denote as attribute-based credentials (ABCs). There are many forms of electronic identity documents (eIDs), such as a driver's license, ID card or a passport. For simplicity we will use the term `passport' for all of these variations. Furthermore, authentication with respect to IRMA cards means the card is presented to a terminal (e.g. in a supermarket) which verifies the card user has the required attributes (e.g. age $\geq$ 18) for purchasing items (e.g. alcohol).

The paper is structured as follows. We start with the theory behind the technologies of IRMA and passports in chapter~\ref{sec:theory}. Following that, in chapter~\ref{sec:goals} we state the goals that the protocol must satisfy and describe several attack scenarios. Afterwards we delve into the specifics of the TLS protocol that the Assurer protocol depends on, followed by the cryptographic assumptions we made with Assurer in mind, and finally the protocol itself. This is found in chapter~\ref{sec:protocols}. Next, in chapter~\ref{sec:formalisation} we describe the formalisation of the protocol as a model in the Pi Calculus for verification using ProVerif and explain the various implementation choices. Finally we discuss our findings and shortcomings and provide a basis for future work in chapter~\ref{sec:conclusion}. The model is listed in appendix~\ref{app:model}.

