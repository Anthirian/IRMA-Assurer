\documentclass[11pt, letterpaper, openany]{memoir}
\usepackage[T1]{fontenc}
%\usepackage[utf8]{inputenc}
\usepackage{fontspec}
\usepackage[pass]{geometry}
\usepackage{graphicx}
\usepackage{tabularx}
%\usepackage{caption}
\usepackage{subfig}
\usepackage{wrapfig}
\usepackage{eso-pic}
%\usepackage{apacite}
%\usepackage{minted}
%\usepackage{listings}
\usepackage{fancyvrb}
%\usepackage[shortcuts]{extdash}
\usepackage{verbatimbox}
%\usepackage[activate={true,nocompatibility},final,tracking=true,kerning=true,spacing=true,factor=1100,stretch=10,shrink=10]{microtype}
\usepackage[english]{babel}
\usepackage[hyphens]{url}
%\usepackage{refcheck}
%\usepackage{cleveref}
\usepackage[colorlinks, linkcolor=black, urlcolor=black, citecolor=black, pdfpagemode=UseNone, pdfstartview=]{hyperref}

\parskip=7pt
\parindent=0pt
%\microtypecontext{spacing=nonfrench}

\newfontfamily\RUfontbold[]{ITCMendozaRoman LT Medium}
\newfontfamily\RUfont[]{ITCMendozaRoman LT Book}

\definecolor{burntorange}{rgb}{0.8, 0.33, 0.0}
\definecolor{darkcyan}{rgb}{0.0, 0.55, 0.55}
\definecolor{darkelectricblue}{rgb}{0.33, 0.41, 0.47}
\definecolor{darklavender}{rgb}{0.45, 0.31, 0.59}
\definecolor{darkmagenta}{rgb}{0.55, 0.0, 0.55}

\newcommand\CenterOnPage[1]{\AddToShipoutPicture*{\AtPageCenter{\makebox(0,0){\includegraphics[width=0.5\paperwidth]{#1}}}}}

%\newenvironment{abstract}%
%{\cleardoublepage\null\vfill\begin{center}%
%\bfseries\abstractname\end{center}}%
%{\vfill\null}

%\captionsetup{format=plain}
%\usepackage{tgcursor}
% ProVerif model style options
\newcommand{\kwl}[1]{\textcolor{blue}{#1}}
\newcommand{\kwf}[1]{#1}
\newcommand{\kwc}[1]{#1}
\newcommand{\kwcom}[1]{\textcolor{gray}{#1}}
\newcommand{\kwp}[1]{\textsf{#1}}
\newcommand{\kwt}[1]{\textsl{\textcolor{darkmagenta}{#1}}}
\newcommand{\kwe}[1]{\textsf{#1}}
\newcommand{\kwtable}[1]{\textsf{#1}}
\newcommand{\var}[1]{#1}

\begin{document}
\frontmatter
\pagestyle{empty}
\newgeometry{margin=2cm}
%\begin{titlepage}
  \CenterOnPage{images/logo}
  \thispagestyle{empty}
  \begin{center}
    \vspace*{2cm}
    \huge{\textsc{\textbf{Radboud University of Nijmegen}}}\\
    \vspace{0.2cm}
    \LARGE{\textsf{Faculty of Science}}\par
    \vspace*{0.5cm}
    \large{\textsf{Master's Thesis Computer Science}}\\
    \LARGE{\textsf{Kerckhoff's Institute}}

    \vspace*{2cm}
    \Huge{\textsf{{IRMA Verified Assurer}}}\\
    \Large{\textsl{Securely storing identity document chip data onto IRMA cards}}

    \vspace*{4cm}
    \begin{tabularx}{0.75\textwidth}{Xr}
      \textsl{Author:} & \textsl{Supervisor:} \\
      G.A. Smelt & prof. dr. B.P.F. Jacobs\\\\
      & \textsl{Reader:}\\
      & dr.ir. E. Poll 
    \end{tabularx}

    \vspace*{2cm}
    \large{\today}
  \end{center}
%\end{titlepage}
\restoregeometry
\pagestyle{plain}

\begin{abstract}
\noindent
Selectively disclosing personal data in order to gain access to some product or service can be achieved by covering the non-relevant parts of one's identity document. However the IRMA implementation is much more secure and privacy-friendly. This requires the person to have previously obtained an identifying set of digital characteristics in order to pass verification by a relying party. IRMA Verified Assurer is an ideal method for new adopters of IRMA technology to obtain an initial set of such attribute-based credentials by leveraging one's identity document, such as a passport. Due to the nature of IRMA such an initialization is required to adhere to strict cryptographic rules. In this paper we describe the theory behind IRMA and passport security. Furthermore we specify guidelines which such a protocol has to follow and we design the Assurer protocol. We prove that the protocol is cryptographically secure by translating it into a model written in the Pi Calculus and proving this model using the ProVerif cryptographic protocol verifier.
\end{abstract}
\clearpage
\pagestyle{headings}
\tableofcontents
\listoffigures

\mainmatter
\section{Introduction}
Consider the scenario where you have to legitmize yourself in order to purchase a particular item such as cigarettes. The cashier is required to check your identification to ensure you are not underage before allowing you to purchase the items. Generally the customer presents some form of identification to the cashier, such as a passport. The cashier then checks your date of birth to ensure you are of legal age. In this process the cashier may also learn among other things your name, location of birth and the document number. To some this feels like a breach of privacy and these individuals may wish to share only their date of birth, but ideally just the property `of legal age' with the cashier. This is what the IRMA card is designed to do. The name is an acronym of I Reveal My Attributes and it is designed in such a way that you can selectively disclose properties of yourself to select individuals. Before you can use such a card it has to be filled with properties about your person, called attributes, which can then be disclosed to gain various privileges, turning the attributes into attribute based credentials or ABCs. A particularly good set of starting properties to use for generating ABCs are the data stored in a person's passport. For this reason, we have designed the IRMA Assurer protocol.

IRMA Assurer is a protocol designed to facilitate transfer of passport information to IRMA cards. In this paper we describe the steps taken during the design process and provide proof it satisfies the goals set out for it. In this protocol we identify three actors: the assurer, the verification and issuing server and the IRMA card. \ldots

Over the course of this paper we use many terms the reader may not be familiar with. Furthermore abstraction is sometimes favorable. This section describes the meaning of certain terms that are used throughout this paper. The IRMA Assurer protocol, denoted `Assurer' from this point onwards, makes use of tablets for the clients that connect to the server. We will simply identify these tablets as `clients' and similarly will refer to the IRMA card simply as `card.' The clients connect to a central server which provides the clients with cryptographically signed attribute data, which we will denote as either attributes, attribute data or attribute-based credentials (ABCs). There are many forms of electronic identity documents (eIDs), such as a driver's license, ID card or a passport. For simplicity we will use the term `passport' for all of these variations. Furthermore, authentication with respect to IRMA cards means the card is presented to a terminal (e.g. in a supermarket) which verifies the card user has the required attributes (e.g. age $\geq$ 18) for purchasing items (e.g. alcohol).

The paper is structured as follows. We start with the theory behind the technologies of IRMA and passports in section~\ref{sec:theory}. Following that, in section~\ref{sec:goals} we state the goals that the protocol must satisfy and describe several attack scenarios. Afterwards we delve into the specifics of the TLS protocol that the Assurer protocol depends on, followed by the cryptographic assumptions we made with IRMA Assurer in mind, and finally the protocol itself. This is found in section~\ref{sec:protocols}. Next, in section~\ref{sec:formalisation} we describe the formalisation of the protocol as a model in the Pi Calculus for verification using ProVerif. Finally we discuss our findings and shortcomings and provide a basis for future work in sections~\ref{sec:conclusion} and~\ref{sec:futurework} respectively. The model is listed in appendix~\ref{app:model}.

\subsection{Motivation}
The IRMA card has a lot of potential. There are many use cases the card could improve upon. Assurer could potentially allow for a much quicker adaptation of the IRMA card. The reason for this is that most people already have an identity card or a passport that contains all their characteristics in one place. Leveraging this in order to quickly gather many attributes at once would be an ideal way to allow newcomers to get started. The passport or identity card generally holds the most generic characteristics, such as name, birthday, height and sex, making its conversion into attributes have immediate visual result.

\section{Theory}
\subsection{IRMA}
Here we explain the background of the IRMA proofs.

\subsection{Passport}
Example: active authentication in e-passport
\begin{itemize}
	\item private key securely embedded in passport chip
  \item public key signed by producer (Morpho in NL)
  \item Morpho's public key signed by Dutch state
\end{itemize}

\chapter{Goals}
\label{sec:goals}
This chapter focuses on the goals of Assurer. We will describe the goals that should be met by the protocol and are going to be verified by means of a protocol prover. First and foremost, the protocol should ensure the clients will be authenticated to the server and vice versa. These properties should prevent susceptibility to man-in-the-middle attacks. Furthermore both the all application data sent between client and server should be secret and trustworthy. Also the protocol should be resistant against replay attacks targeting ABCs. Finally the protocol should be designed in such a way that the connection to the server will not cause a denial of service.

To summarize the protocol should offer:
\begin{itemize}
  \item authentication of client to server;
  \item authentication of server to client;
  \item integrity of passport data sent between server and client;
  \item confidentiality of passport data sent between server and client;
  \item integrity of ABCs sent between server, client and card;
  \item confidentiality of ABCs sent between server, client and card;
  \item resistance against replay attacks.
\end{itemize}

In addition, we wish to achieve perfect forward secrecy, meaning an attacker cannot derive previous session keys even if private keys are obtained (see chapter~\ref{subsubsec:pfs}). This means we have to use cryptographically strong keys, which also should be ephemeral.

Note that replay attacks are partly mitigated by policies surrounding the system. For example, replaying of ABCs is only possible should an attacker have access to the assurer's tablet, which is kept in a safe and is inaccessible. To improve upon this, implementations of the protocol should ensure it does not allow or facilitate storing ABCs internally for repeated use. All ABCs should be stored on the corresponding IRMA card and securely deleted immediately after.

\section{Attack scenarios}
This section describes possible attack scenarios with respect to the goals described above. These are attacks the protocol should be resistant against. The possible impact of such an attack is described directly after the attack itself, along with the likelihood of such an attack occurring.

\begin{itemize}
	\item An attacker reads the passport data sent between client and server. This attack causes personally identifiable information to be learned, leading to a breach in confidentiality. This may be likely in the event weak cryptography is used or if either party loses its key. However because of the requirement of perfect forward secrecy this issue is partly mitigated (see section~\ref{subsubsec:pfs}).
  \item An attacker modifies the passport data sent between client and server. This attack causes the server to receive data inconsistent with the passport data sent by the client, at worst leading to the issuance of ABCs not corresponding to the passport. This can be mitigated, like the previous scenario, with the use of sufficiently strong cryptography, but is still somewhat likely to occur with a powerful adversary.
  \item An attacker modifies signed ABCs. This attack causes the client to receive incorrect ABCs, at worst leading to the issuance of ABCs not corresponding to the passport. This is highly unlikely as it would require the attacker to be able to sign the ABCs using the issuer's private key to keep relying parties from detecting the fraud.
  \item An attacker intercepts ABCs and stores it onto his own card. This attack causes false issuance of ABCs, in turn leading to possible fraud. This may be a likely scenario if the attacker controls the network and is actively attacking the protocol. A mitigating factor is the fact the attacker must first obtain access to the tablet, which as described is kept under lock and key.
  \item An attacker submits forged passport data to the server. This attack causes the server to possibly provide ABCs for fictitous people, in turn leading to possible fraud. Because the server verifies all passport data before providing ABCs, it may also be possible to query the server, guessing correct personally identifiyable information.
  \item An attacker intercepts and later reuses ABCs already stored onto a card for storing on another card. This attack causes the attacker to be able to commit fraud. \textsc{TODO: Likely?}
  \item An attacker replays the sending of passport data to obtain another identical set of ABCs. This attack effectively facilitates the cloning of cards. Since the server is keeping a log of all of the ABCs that are given out a simple lookup would reveal the replay attack, making this an unlikely scenario.
  \item An attacker uses different ABCs from different people to authenticate. This attack allows combining ABCs to achieve the required combination for authentication. This is hardly an issue, due to the fact the IRMA card features a photo that should resemble the authenticating person. Adding to that the fact that switching cards midway would raise eyebrows leads us to think this attack is unlikely to occur.
  \item An attacker causes a denial of service on the client. This may cause clients to not receive the requested ABCs, causing them to restart the process. This does not lead to a security risk.
  \item An attacker causes a denial of service on the server. This halts all activity surrounding the issuance of ABCs, but does not lead to a security risk.
  \item An attacker shows up with someone else's passport. At worst this could cause incorrect ABCs to be placed onto an IRMA card, however this is mitigated by verifying the photograph printed on the passport. In order to pass this test the attacker has to forge the passport, which is detected upon verification by either the client or the server. This scenario therefore is unlikely.
  \item An attacker switches out the non-initialized IRMA cards before they reach the assurer's office. This also is a non-issue, because these cards only contain a unique ID at this point in the process. This ID is not used by the relying parties for verification of ABCs and therefore has no impact on the security and privacy of the system.
\end{itemize}

\subsection{Perfect Forward Secrecy}
\label{subsubsec:pfs}
In key exchange protocols there is a property called Perfect Forward Secrecy (PFS) that provides a more secure way of encryption with respect to everyday session encryption, because the key is deleted immediately after use and therefore cannot be stolen by an attacker or forced to be handed over by the government in an attempt to decrypt intercepted traffic~\cite{lecture2}. More specifically, the exposure of long-term keying material, used in the protocol to negotiate session keys, does not compromise the secrecy of session keys established before the exposure. This property is especially relevant to scenarios in which exchanged session keys require secrecy protection beyond their lifetime, such as in the case of session keys used for data encryption. This means it is relevant in Assurer, since a lot of sensitive information is being transfered and therefore needs to be kept secret.

The most common way to achieve PFS in a key-exchange protocol is by using the Diffie-Hellman key agreement with ephemeral exponents to establish the value of a session key, while confining the use of the longterm keys (such as private signature keys) to the purpose of authenticating the exchange (see authentication). One essential element for achieving PFS with the Diffie-Hellman exchange is the use of ephemeral exponents which are erased from memory as soon as the exchange is complete. This should include the erasure of any other information from which the value of these exponents can be derived such as the state of a pseudo-random generator used to compute these exponents~\cite{PFS}.

\section{Cryptographic properties}
To summarize the previous sections, the protocol should satisfy the following cryptographic properties.
\begin{itemize}
  \item Authenticity
  \item Accountability
  \item Confidentiality
  \item Integrity
  \item \scriptsize Availability
\end{itemize}


\section{Protocols}
The IRMA Assurer protocol is an application level protocol. It is designed to run on top of TLS 1.2. Only the latest version of the TLS standard is considered cryptographically secure and therefore is the only version IRMA Assurer supports. A second argument for the use of version 1.2 is the possibility to achieve Perfect Forward Secrecy (PFS), meaning that in case the long term key is ever compromised, then the session keys derived from it before compromise are still secure~\cite{PFS}. This chapter will first describe the TLS protocol, explaining the chosen parameters. The IRMA Assurer protocol follows directly after.

\subsection{Transport Layer Security}
The Transport Layer Security (TLS) protocol is a protocol operating on the presentation layer of the OSI Reference Model~\cite{osi}. It is a protocol that secures the connection between two parties across an insecure channel, ensuring secrecy~\cite{tls1.2}. It also has the possibility for authentication based on certificates. TLS is commonly used in web browsers to encrypt HTTP into HTTPS traffic, but it is capable of protecting any TCP connection~\cite{lecture}.

TLS 1.2 was defined in RFC 5246 in August 2008. It is based on the earlier TLS 1.1 specification. It was further refined in RFC 6176 in March 2011 removing their backward compatibility with SSL such that TLS sessions will never negotiate the use of Secure Sockets Layer (SSL) version 2.0.

\texttt{TLS vulnerable to handshake renegotiation attack, solved by RFC 5746~\cite{lecture}.}

The protocol consists of two major parts, TLS Handshake and TLS Record. Handshake is used for setting up connections between two parties, with optional authentication, while Record is used for ordering, encrypting and sending of the data. Handshake makes use of Record, so they work both alongside as well as on top of each other (see figure~\ref{fig:tlsstack}).

\begin{figure}[htb]
	\centering
		\includegraphics[width=0.40\textwidth]{images/tlsstack.png}
	\caption{TLS Protocol Hierarchy}
	\label{fig:tlsstack}
\end{figure}

\subsubsection{TLS handshake}
Figure~\ref{fig:tlshandshake} shows a schematic overview of the TLS handshake, copied from RFC 5246~\cite{tls1.2}. The \texttt{*} indicates a step necessary for client-to-server authentication. The handshake begins when a client connects to a TLS-enabled server requesting a secure connection and presents a list of supported cipher suites (ciphers and hash functions). From this list, the server picks a cipher and hash function that it also supports and notifies the client of the decision. The server usually then sends back its identification in the form of a digital certificate. The certificate usually contains the server name, the trusted certificate authority (CA) and the server's public encryption key. The client may contact the server that issued the certificate (the trusted CA) and confirm the validity of the certificate before proceeding. In order to generate the session keys used for the secure connection, the client encrypts a random number with the server's public key and sends the result to the server. Only the server should be able to decrypt it, with its private key. From the random number, both parties generate a 'master secret' and then negotiate a session key for encryption and decryption.

The details about the underlying cryptographic functions selected for the IRMA Assurer protocol, such as encryption and hashing functions will be discussed later in section~\ref{sec:assumptions}.
\begin{figure}[htb]
\centering
\begin{verbatim}
        Client                                               Server

        ClientHello                  -------->
                                                        ServerHello
                                                       Certificate*
                                                 ServerKeyExchange*
                                                CertificateRequest*
                                     <--------      ServerHelloDone
        Certificate*
        ClientKeyExchange
        CertificateVerify*
        [ChangeCipherSpec]
        Finished                     -------->
                                                 [ChangeCipherSpec]
                                     <--------             Finished
        Application Data             <------->     Application Data
\end{verbatim}
\label{fig:tlshandshake}
\caption{TLS Handshake}
\end{figure}

\subsubsection{TLS record}
The record protocol handles the sending and receiving of TLS related messages. It forms the basis of the TLS protocol. When sending, it will split the data in blocks, optionally compress it, apply a MAC, encrypt the data, add a fragment header and finally send the data over TCP port 443. When receiving, it will decrypt the data, verify the MAC, optionally decompress, defragment and finally deliver the data to the upper layer. Essentially this protocol forms the secure channel between client and server~\cite{tls1.2}.

\subsection{IRMA Assurer}
The IRMA Assurer protocol makes use of TLS 1.2 as described above. This means both the server and the tablets will verify each other's certificates and agree upon a session key. Once a secure channel has been successfully set up, we have achieved privacy and data integrity of all communication over this channel~\cite{tls1.2}.

\subsection{Assumptions}
\label{sec:assumptions}
In this section we discuss the assumptions made with regard to Assurer. The section is divided into two parts. The first section discusses assumptions on the operational level and the second describes assumptions with respect to cryptography.

\subsubsection{Operational}
Operational assumptions mostly focus on procedures that have to be followed for the Assurer protocol to work. We reiterate some parts of the basic course of events described in section~\ref{subsec:bcoe} to make them explicit as assumptions. 

The entire ecosystem in which Assurer operates can be described as a star structure. At the center of the star will be the (only) server. The edges (points) of the star are formed by the clients. Clients do not communicate with each other, but only with the server. Each of these clients has their own NFC-enabled tablet, which is locked up in a safe and PIN code protected to prevent malicious use. By using an Android app installed on these tablets assurers may access data on passport chips and IRMA cards. Upon client initialization this app is installed on the tablet, as well as the DNS host name of the server. By using the host name the server has the option to switch IP addresses without breaking the system. Also installed on the tablet are a client certificate (signed by the server) required for authentication and its corresponding private key. The server checks the certificate for validity and also checks if it has not been revoked by accessing Certificate Revocation Lists (CRL), which mitigates issues with stolen tablets or misuse. Finally the public key of the server is installed for easy access.

Although subject to change, at the time of writing the communication between IRMA card and terminal, i.e. tablet, is still being sent in the clear. Upon reading data from a passport chip the integrity is verified by the client by executing BAC and PA, and sent to the server for further analysis. The server performs the same checks, plus AA, and looks up the person corresponding to the data in a central database. If everything is proven to be valid the server will create digitally signed attribute-based credentials that correspond to the passport data. The server will then send these ABCs to the client, who will in turn install them on the person's IRMA card. Upon successful installation onto the IRMA card, the ABCs are securely deleted from both server and client. Furthermore, the passport data is securely deleted from the server and client. However the server does keep a log of the BSN that was part of the attribute signing request sent by the client, along with a timestamp. This facilitates traceability in case of anomalies. Finally, we do not support TLS session resuming. This property allows clients to send a stored session identifier to the server and then pick up where left off. Supporting this weakens the strength of the TLS connection, in particular the PFS property of TLS.\footnote{\url{https://timtaubert.de/blog/2014/11/the-sad-state-of-server-side-tls-session-resumption-implementations/}} Since we wish to have PFS we do not allow session resuming.

\subsubsection{Cryptographic}
Before going into the cryptographic choices regarding Assurer it is important to take a quick look at the transport layer used below. There are two options, the User Datagram Protocol (UDP) and the Transport Control Protocol (TCP). We have chosen to make use of TCP for its transport layer protocol. The main reason for this is that we require the packets to be delivered without packet loss. In other words, we favor TCP over UDP for its high reliability. Furthermore it is best compatible with the TLS protocol which we aim to use. Other arguments for choosing TCP include its data flow control and packet reordering capabilities.

In order to achieve the goals described in section~\ref{sec:goals} several choices have been made regarding the cryptography. What follows is an argumentation for the decisions made for Assurer. 

The guideline for choosing supported cipher suites is to offer perfect forward secrecy. Any cipher suites that offer PFS capabilities have been selected, while suites that do not are not selected. In principle, any public key encryption scheme can be used to build a key exchange with PFS by using the encryption scheme with ephemeral public and private keys~\cite{PFS}. This means we have the option to use any public key infrastructure as long as we throw away the keys when we're done using them. Furthermore we can either choose elliptic curve cryptography or conventional cryptography. Because we wish to achieve mutual authentication both parties need to exchange certificates. These certificates must be of type X.509v3 in accordance with the TLS specifications~\cite{tls1.2}. The type of cryptography used for these certificates may either be RSA or DSA. Generally, in terms of performance, neither is significantly better than the other, meaning we may support both. 

As explained above we have chosen TLS 1.2 for its resistance against publicly known feasible attacks, as well as for its Perfect Forward Secrecy capabilities. This requires that the key generation for the encryption scheme must be fast enough. For most applications this disqualifies, for example, the use of ephemeral RSA public key encryption for achieving PFS, since the latter requires the generation of two long prime numbers for each exchange, a relatively costly operation~\cite{PFS}. For this reason we have selected the Diffie-Hellman Key Exchange protocol. Furthermore because of the faster key generation, better performance and shorter key length while still achieving the same level of security we have chosen to use elliptic curves.\footnote{\url{https://www.nsa.gov/business/programs/elliptic_curve.shtml}}

Since the Assurer protocol is to be used only on newly developed hardware and software it is safe to use the newest cryptography; there should not be any compatibility issues. Mozilla lists the following as the best choice for modern clients in their documentation on server-side TLS,\footnote{\url{https://wiki.mozilla.org/Security/Server_Side_TLS}} ordered from most recommended to least recommended (the $!$ indicates forbidden):

\begin{description}
	\item [Ciphersuite] \texttt{ECDHE-RSA-AES128-GCM-SHA256, ECDHE-ECDSA-AES128-GCM-SHA256, \\
ECDHE-RSA-AES256-GCM-SHA384, ECDHE-ECDSA-AES256-GCM-SHA384, \\
DHE-RSA-AES128-GCM-SHA256, DHE-DSS-AES128-GCM-SHA256, kEDH+AESGCM, \\
ECDHE-RSA-AES128-SHA256, ECDHE-ECDSA-AES128-SHA256, \\
ECDHE-RSA-AES128-SHA, ECDHE-ECDSA-AES128-SHA, ECDHE-RSA-AES256-SHA384, \\
ECDHE-ECDSA-AES256-SHA384, ECDHE-RSA-AES256-SHA, ECDHE-ECDSA-AES256-SHA, \\
DHE-RSA-AES128-SHA256, DHE-RSA-AES128-SHA, DHE-DSS-AES128-SHA256, \\
DHE-RSA-AES256-SHA256, DHE-DSS-AES256-SHA, DHE-RSA-AES256-SHA, \\
!aNULL, !eNULL, !EXPORT, !DES, !RC4, !3DES, !MD5, !PSK}
  \item [Versions] \texttt{TLSv1.1, TLSv1.2}
  \item [RSA key size] \texttt{2048}
  \item [DH Parameter size] \texttt{2048}
  \item [Elliptic curves] \texttt{secp256r1, secp384r1, secp521r1} (at a minimum)
  \item [Certificate signature] \texttt{SHA-256}
  \item [HSTS] \texttt{max-age=15724800}
\end{description}

Note that this is the optimal configuration for Mozilla's own servers providing HTTPS connections, which explains the mention of HSTS. For the Assurer protocol we are only interested in cipher suites supported in TLS 1.2, while making use of elliptic curves and Diffie-Hellman, meaning we can safely ignore anything that is unrelated. Cross-referencing this list of cipher suites with the list of elliptic curve and TLS 1.2 cipher suites described by the OpenSSL documentation\footnote{\url{https://www.openssl.org/docs/apps/ciphers.html}} yields the following list of cipher suites supported by the IRMA Assurer protocol, in descending order of priority, named following IANA guidelines. This naming convention differs from the one used by Mozilla, who use OpenSSL guidelines, but is in correspondence with the RFC documents on TLS. The blank lines indicate Mozilla favors either a non-elliptic curve or a TLS 1.2 incompatible cipher suite over the ones that follow. 

\begin{verbatim}
TLS_ECDHE_RSA_WITH_AES_128_GCM_SHA256
TLS_ECDHE_ECDSA_WITH_AES_128_GCM_SHA256
TLS_ECDHE_RSA_WITH_AES_256_GCM_SHA384
TLS_ECDHE_ECDSA_WITH_AES_256_GCM_SHA384

TLS_ECDHE_RSA_WITH_AES_128_CBC_SHA256
TLS_ECDHE_ECDSA_WITH_AES_128_CBC_SHA256

TLS_ECDHE_RSA_WITH_AES_256_CBC_SHA384
TLS_ECDHE_ECDSA_WITH_AES_256_CBC_SHA384
\end{verbatim}

There are many reasons for this particular ordering. We highlight the main reasons below. 

\begin{itemize}
  \item ECDHE+AESGCM ciphers are selected first. These are TLS 1.2 ciphers and not widely supported at the moment. No known attack currently targets these ciphers.
  \item PFS ciphersuites are preferred, with ECDHE first, then DHE.
	\item ECDHE provides faster handshakes than DHE.\footnote{\url{http://vincent.bernat.im/en/blog/2011-ssl-perfect-forward-secrecy.html}}\,\footnote{\url{http://nmav.gnutls.org/2011/12/price-to-pay-for-perfect-forward.html}}
  \item AES 128 is preferred to AES 256, because it provides good security, is really fast, and seems to be more resistant to timing attacks.
  \item SHA256 is favored over SHA384. This appears to be mainly for interoperability purposes.
\end{itemize}

Galois Counter Mode (GCM) is an Authenticated Encryption (AE) algorithm. AE algorithms are designed to provide both data authenticity (integrity) and confidentiality. Since both are goals of Assurer we favor GCM over CBC, but do not require it.

For more information about the reasons behind this ordering please see~\cite{mozilla}.

\subsubsection{DH parameters}
Unfortunately, some widely used clients lack support for ECDHE and must then rely on DHE to provide perfect forward secrecy. This is true for Android $< 3.0.0$, Java $< 7$ and OpenSSL $< 1.0.0$, among others. Nowadays many, if not all, devices run Android versions higher than 3.0.0, so we don't expect to see issues here. However the Java requirement could lead to issues when implementing Assurer as an Android app, mainly because of the many IRMA-related dependencies that such an app has to deal with. These dependencies may not support Java 7 or 8 yet. Adding to that, Java 6 and 7 do not support Diffie-Hellman parameters larger than 1024 bits. This has consequences for the PFS requirement of Assurer. The only way a secure connection can be achieved from a Java 6 client is to use DHE cipher suites and use 1024-bit groups.

Use of the most widely used 1024-bit pre-computed Oakley group 2, standardized in~\cite{ike}, is considered unsafe, mainly because it is very likely that a state-level adversary may have broken it. In any case it is recommended to generate a random DH group instead of using a standardized one when setting up a new server~\cite{logjam}.

CBC ciphers can be attacked with the Lucky Thirteen attack\footnote{\url{http://www.isg.rhul.ac.uk/tls/Lucky13.html}} if the library is not written carefully to eliminate timing side channels. This attack requires multiple sessions and is possibly detectable due to the low volume of traffic in the IRMA Assurer protocol.

\paragraph{Logjam}
When choosing this list of supported cipher suites we have chosen the ``modern'' preset described by Mozilla. This is especially important, because of the recently discovered Logjam attack on the Diffie-Hellman key agreement, in particular within TLS, SSH and VPN connections~\cite{logjam}. Essentially Logjam is a type of attack that allows an attacker to downgrade the security of the connection to DHE export cipher suites. It is also possible to attack weak ($\leq$ 1024-bit) Diffie-Hellman groups by precomputing the group and then using it for lookups. The short term solution for mitigating Logjam attacks is for servers to disable export ciphers and use freshly generated groups of 2048 bits or larger. Clients should no longer accept groups of sizes lower than 1024 bits. In the long term however it is preferable to switch to elliptic curve cryptography (ECC) and not allow any other cipher suites. This is because none of the attacks presented work against ECC. As described above the Assurer protocol only makes use of a subset of the ``modern'' preset, which in turn only makes use of ECC cipher suites, and therefore is not susceptible to the Logjam attack.

\paragraph{Elliptic curves}
NIST has defined 15 standard curves.\footnote{\url{http://nvlpubs.nist.gov/nistpubs/FIPS/NIST.FIPS.186-4.pdf}} However, in practice, many implementations only support two of them, P-256 and P-384, because that's what the NSA recommends. As seen above Mozilla recommends to use at the very least P-256, P-384 and P-521. For Assurer we will follow that recommendation.

\subsubsection{Certificates}
\begin{itemize}
	\item Each client posesses one certificate, signed by the server, with which it may authenticate to the server. This occurs during the run of the TLS handshake protocol.
  \item Each passport contains one certificate with which the authenticity may be verified (see section \ldots
\end{itemize}

\subsubsection{Keys}
The server owns two keypairs. The first keypair is used for signing and verification of certificates while the second one is used for the issuing and verification of attribute based credentials. A client only owns one keypair, which is used for signing and verification of certificates. The server's public key used for signing and verification is pre-loaded onto the clients during initialization, while the clients' public keys are stored within the pre-loaded certificates.

The use of TLS 1.2 with the aforementioned cryptographic options for the handshake will result in a symmetric session key held by both parties. From this point onwards we therefore have a secure channel to use for application data transfer.

\subsubsection{Application data}
All application data is protected with the session key as generated by the TLS handshake protocol. 

\ldots


\subsubsection{Final remarks}
\begin{itemize}
  \item 
\end{itemize}



\section{Formalisation}
\label{sec:formalisation}
In this section we describe the formalisation of the assumptions and use cases into a model written in the Pi Calculus. This model can be proven to be cryptographically secure using ProVerif. We have chosen ProVerif for its ability to automatically analyze the security of cryptographic protocols. ProVerif is capable of proving reachability properties, correspondence assertions, and observational equivalence. These capabilities are particularly useful to the computer security domain since they permit the analysis of secrecy and authentication properties. Moreover, emerging properties such as privacy, traceability, and verifiability can also be considered. Protocol analysis is considered with respect to an unbounded number of sessions and an unbounded message space. Moreover, the tool is capable of attack reconstruction: when a property cannot be proved, ProVerif tries to reconstruct an execution trace that falsifies the desired property~\cite{proverifmanual}.

The model we have constructed for Assurer is listed in appendix~\ref{app:model}. This model is based on the TLS handshake model by Tankink and Vullers~\cite{tankink2008verification}. Their model makes use of RSA for the key exchange, but our model uses Diffie-Hellman. Similarly to their model we make use of message tagging, as is generally considered good practice. Furthermore, ProVerif has trouble finding attacks on variables that are being computed instead of being declared. Tankink and Vullers solve this by creating a new flag variable and outputting this on the supposed-to-be secret. This way when an attacker is able to obtain the flag, then he must have knowledge of the secret and an attack trace can be found. Furthermore we have added dead code checks to the model. This helps determine if the model fails midway. Such a failure is indicated by the fact that no falsification is found by ProVerif. Finally ProVerif appears to have a set a limit on the amount of RAM that it may use. It will not use more than 2GB of RAM, even though plenty is still available, and will consequently report a fatal error. To work around this, we have chosen to split the model into two parts: the first part describes the TLS handshake, while the second part describes the application protocol.

The model consists of three distinct phases. During the first phase the TLS handshake protocol is executed. As explained before we make use of an ephemeral Elliptic Curve Diffie-Hellman (ECDHE) key agreement. This means that in this protocol the only use for both parties' keypair is the signing and verification of messages sent during the key agreement. Because of the nature of Diffie-Hellman, i.e. the Discrete Logarith Problem (DLP), it is not required to encrypt the protocol parameters as they can be sent in the clear without risk of an eavesdropper learning the agreed upon key. For authentication purposes however, we do require these protocol parameters to be signed by the sending party. This allows the receiving party to verify the identity of the sender before starting an encrypted session~\cite{tls1.2,tlsecc}.

The second phase starts the actual Assurer application protocol, which takes place over the encrypted channel for which the key was negotiated during the TLS handshake of phase one. During this phase the passport is verified on the client's side by using passive authentication. This is done by checking the hash stored in the Security Object file of all the data groups contained within the passport. Once the client is confident that the integrity of the passport holds it proceeds by sending the passport data, plus a fresh nonce, to the server via the encrypted channel. The server will then also perform passive authentication as a double-check. If the server agrees with the client on the integrity of the passport, active authentication is performed. To do this, it sends a challenge that the passport has to solve. The passport has to sign this challenge in order to prove it has knowledge of the embedded private key. During passive authentication the server has obtained the passport's public key, which is used to verify the response to the challenge.

The third and final phase of the protocol is where the issuing of the Attribute Based Credentials takes place. As mentioned before the server is the only party that posesses the issuer private key. Every verifier, i.e. a terminal, for example at a supermarket has knowledge of the issuer's public key and is therefore able to verify the signature placed on the ABCs. The issuer's public key is not used in this model, as attribute verification is not included in its scope. The server creates ABCs and encrypts these ABCs, along with the nonce sent by the client during phase two as well as both parties' names, using the session key and computes an HMAC over the result. The encrypted data and its HMAC are sent to the client, which in turn verifies the HMAC and decrypts the data. If the decrypted data is found to match the nonce and the names of both parties, the client accepts the ABCs and stores them onto the IRMA card and closes the session.

\subsection{Discussion}
Here we reflect on the details of the model. This reflection is separated into two parts: the handshake part and the Assurer part. For each part we first describe the different functions we use throughout the model, followed by a description of the queries and finally the parties involved. Should the reader not have experience with ProVerif, we would recommend familiarizing oneself with the basics~\cite{proverifmanual} in order to better understand this section.

\subsubsection{Handshake}
The functions we use for this model are fairly straightforward, apart from the key derivation functions. First of all is a hash function which used for both verification of the client's signature and integrity of the \texttt{finished} messages. We also have encrypt and decrypt functions, which are symmetric and use the generated session key. This model does not use assymmetric cryptography for encryption and decryption. It does however use keypairs and certificates, generated by the \texttt{keypair} and \texttt{cert} function, but these are used for signing and verification of the Diffie-Hellman parameters sent by the server. The public and private parts of a keypair are obtained using the \texttt{pk} and \texttt{sk} functions respectively. It is also important to note that the \texttt{verify} function does not actually verify a message that was signed, but rather retrieves the public key that is included in the certificate. For actually removing the signature and verifying the message the \texttt{unsign} function is used.

Furthermore we have a pseudo-random number function (\texttt{PRF}) which results in an initialization vector when provided with the master secret. With this initialization vector and the functions \texttt{clientK} and \texttt{serverK} the client and server derive the session key respectively. 

Finally we have the Diffie-Hellman functions \texttt{G} and \texttt{sm}. The \texttt{G} function is actually not a function, but a constant instead and represents the generator point of the finite cyclic group of points on the elliptic curve. The \texttt{sm} function allows scalar multiplication. The modulus is abstracted from in this computation, as it is implied and proves to be a large challenge in Pi Calculus. Using the equation defines the relation between the generator point of the group and the coefficients, and results in a usable key agreement scheme.

The query section describes the goals that must be met after checking it with ProVerif. These directly reflect the goals we set in section~\ref{sec:goals}. For the model of TLS handshake we state that an attacker may not learn \texttt{Sa} and \texttt{Sb}, which are flags that both \texttt{A} and \texttt{B} output on the newly generated secure channel \texttt{s}, which is encrypted using the key agreed upon during the execution of the handshake.

The next set of queries state that an attacker may not learn \texttt{PMSa}, \texttt{PMSb}, \texttt{MSa} and \texttt{MSb}. These are four flags, half of which are output by both \texttt{A} and \texttt{B} within the model to allow ProVerif to check that the attacker cannot learn the Pre-Master Secret (\texttt{PMS}) and the Master Secret (\texttt{M}), respectively. These constraints ensure an attacker cannot decrypt the traffic sent over the encrypted channel.

Furthermore we check for the secrecy of the \texttt{Finished} messages sent at the end of the handshake by both parties. These messages are already encrypted with the generated session key and for an attacker to have knowledge of the plaintext would mean the channel is not secure.

Next we check the authentication status for both client and server, as we require mutual authentication. This is achieved by ensuring the events \texttt{endServerAuth} and \texttt{endClientAuth} must always be preceded by the events \texttt{beginServerAuth} and \texttt{beginClientAuth} respectively. These are injective queries, which means that all \texttt{end}-events must be preceded by exactly one \texttt{begin}-event, but it is not required that all \texttt{begin}-events lead to an \texttt{end}-event. For the queries in our model this means that whenever we observe an \texttt{end}-event, and thus assume a party to have authenticated to another, then there must have been a session in which the other party has generated a \texttt{begin}-event. As stated previously, this satisfies mutual authentication between both parties when both queries hold.

Finally, we use two queries to ensure the entire model correctly executes. Essentially the output of \texttt{serverFinished} and \texttt{clientFinished} on a public channel serves as a dead code check. This query must hold, for if it does not then none of the other queries can be considered to have been proved succesfully, since the model has not been verified fully.

Both the \texttt{server} and the \texttt{client} process have been designed to closely resemble the specifications of RFC 5246~\cite{tls1.2}. Differences include message tagging, as mentioned before, for easier reading and ensuring correct execution of the model. The \texttt{server} process also features replication, which allows for multiple parallel runs of the process, in turn allowing the server to accept sessions from multiple clients. This represents the star-architecture of the client-server connections. Note that a \texttt{client} process will always initiate the TLS handshake, since it is the client who requests passport data to be turned into ABCs. Another important difference from the speficiation is that at the end both processes output the flags mentioned above onto a public channel to allow for secrecy and dead code checks.

Apart from the aforementioned two processes there is also the \texttt{initializer} process that is not part of the TLS specification. This process handles the task of setting up keypairs and certificates for both client and server. This initialization takes place on separate (secure) channels, to mimic the pre-loading of certificates and keypairs that takes place when a person applies for a new IRMA card as well as the first (and only) time the server is being set up. To ensure ProVerif has all available non-secret parameters we let the \texttt{initializer} process publish those to a public channel, otherwise some attacks may not be discovered.

\subsubsection{Assurer}
The model of Assurer builds upon the precedents set by the model of the handshake discussed above. Because of the limits set on the RAM usage explained above we have split the model into two parts. The Assurer part therefore has to ensure the properties proven by the handshake part are still true. If this is not the case, the proof of the Assurer part will not hold once the entire model is verified, should ProVerif ever be improved to allow for more RAM to be used. The way we have chosen to circumvent this issue is by keeping the \texttt{client} and \texttt{server} processes intact, but instead of performing the handshake first, we assume we already know about the secret channel \texttt{s} and use that for further communication.

Another change from the handshake model is the fact that the \texttt{initializer} process now also initializes the \texttt{server} process with a secondary keypair used for issuing of ABCs. This issuer keypair is sent over yet another private channel, which means the server cannot mistake one for the other. Also the initializer now generates the session key instead of having both parties agree on one. This is another design choice we have made, since the key agreement has already been proven to be secure in the first part of the model. An attacker thus effectively cannot distinguish the session key in this part of the model from the one agreed upon during the handshake part. 

With this second part of the model also comes a new function. The \texttt{hmac} function is added to ensure the integrity of both passport data and ABCs is kept. We do add any other new functions to our model, but we make use of the functions already present.

The second part of the model also adds several new queries to be proved by ProVerif. First there are two new flags \texttt{passportFlag} and \texttt{abcFlag}, which are used to prove secrecy of the sent passport data and ABCs respectively.

The Assurer model adds new injective queries as well. These injective queries are defined to prove Passive Authentication and Active Authentication for the passport data, as well as ensure no ABCs may ever be sent without completing all required checks beforehand.

Finally we use the same methodology of checking for dead code issues as we did in the handshake model.

The system shows the \texttt{client} process creates a passport object from a DataGroup, which is the internal representation of a passport's data. There are 16 of these DataGroups in total, one of which (DG15) is used for storing the public key of the passport. In our model we have simplified this to a single DataGroup and have created a separate variable to contain the keypair, the public part of which is otherwise stored in DG15. These abstractions are simply to allow for less variables and thus faster verification, without impacting the security proofs. The passport sent to the server therefore contains the DataGroups, a hash of these DataGroups called the Security Object (SOD file) and the public key. Also sent along are both parties' names and a nonce. This passport is protected by an HMAC generated using the session key.

Upon receiving the passport the server checks the hash and thus performs Passive Authentication (PA). Only after PA is performed will the server proceed with Active Authentication. This simply involves the server generating a nonce (challenge) the passport has to sign (response) to prove it has the secret key corresponding to the public key within the passport. 

After both PA and AA are complete the server proceeds by creating ABCs. In the model it creates a single characteristic named \texttt{Char} and signs it using the issuer key. The attribute data is then encrypted along with both parties' names and the previously received nonce. This is then sent to the client along with another HMAC generated using the session key. Upon receiving this the client checks the nonce and the names. At the end of both processes the parties output another flag to verify an attacker cannot obtain the passport data or the ABCs, as well as output the dead code check. Once again, if the dead code check does not pass, the proof of all other queries does not hold.


\chapter{Conclusion}
\label{sec:conclusion}

\ldots

\chapter{Future work}
\label{sec:futurework}

Client moet dienen als doorgeefluik voor attributen en dus niet kunnen zien welke hij op de kaart plaatst. Hiertoe maken we een niet-transparante tunnel van server naar IRMA kaart.

\section{Implementation}
Denk aan
\begin{itemize}
	\item Bouncy Castle voor het gebruik van TLS 1.2, of anders JSSE. JSSE zit standaard in elke Java distro. Probleem met Bouncy Castle is dat het mogelijk niet hoger kan dan TLS 1.0~\cite{sslanalysis} (hoofdstuk 4). 
\end{itemize}

I have gotten started on the implementation of this protocol in Java. The current version makes use of Netty for the communication. This appeared to be an efficient method of implementing communication between two parties. However since I was fully unfamiliar with the framework, I may not have implemented it correctly.

\texttt{TODO: Uitleg waarom Netty}

The implementation is not at all finished and still needs lots of work. At the time of writing the only parts that are finished is setting up an insecure communication between a client and a server, reading the data from a passport and converting it into bits for transport. Because these are only the basic elements of the protocol, one may desire to stop using Netty altogether.


\appendix
\section{Model}
\label{app:model}

\subsection{TLS handshake}
%\input{tls_model}

\subsection{Application data}
%\input{app_model}

\backmatter
\nocite{*}
\bibliographystyle{ieeetr}
\bibliography{references}


\end{document}
